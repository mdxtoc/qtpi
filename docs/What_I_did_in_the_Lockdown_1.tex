%!TEX TS-program = pdflatexmk -xelatex
\documentclass[11pt,a4paper]{article}
\usepackage{times}
\usepackage{graphicx}
\usepackage{amsmath}
\usepackage{amssymb}
\usepackage{subfigure}
\usepackage[round]{natbib}
\usepackage[pagebackref]{hyperref}

\usepackage{fontspec}
\setmainfont{Times New Roman}

\newcommand{\stix}[1]{{\fontspec{STIXGeneral-Regular}#1}}
\newcommand{\stixi}[1]{{\fontspec{STIXGeneral-Italic}#1}}
\newcommand{\Symbol}[1]{{\fontspec{Apple Symbols}#1}}

\hoffset = 0mm
\voffset = 0mm
\textwidth = 165mm
\textheight = 229mm
\oddsidemargin = 0.0 in
\evensidemargin = 0.0 in
\topmargin = 0.0 in
\headheight = 0.0 in
\headsep = 0.0 in
\parskip = 3pt
\parindent = 0.0in

\makeatletter
\renewcommand{\@makefntext}[1]{\setlength{\parindent}{0pt}%
\begin{list}{}{\setlength{\labelwidth}{1em}%
  \setlength{\leftmargin}{\labelwidth}%
  \setlength{\labelsep}{3pt}\setlength{\itemsep}{0pt}%
  \setlength{\parsep}{0pt}\setlength{\topsep}{0pt}%
  \footnotesize}\item[\hfill\@makefnmark]#1%
\end{list}}
\makeatother

%\newcommand{\greyorcolour}[2]{#2} % #1 highlighting (etc.) for greyscale, #2 for colour
%\input{rbmacros}

\title{What I did to Qtpi in the Lockdown (part 1)}
\author{Richard Bornat \\ School of Science and Technology, Middlesex University, London, UK \\ R.Bornat@mdx.ac.uk}

\begin{document}
\maketitle

I'm well over 75, it's about seven weeks into my personal lockdown, and my main distraction so far has been hacking Qtpi. I've done several things: unicoding, overloading operators, qbit collections, iteration, and sparse vectors and matrices.

\section{Unicoding}

Some of the operators in Qtpi are pretty ugly in their ASCII form, so I switched to a lexer which can do Unicode. Here's what has happened so far

\begin{table}
\caption{Unicode symbols}
\centering
\begin{tabular}{|l|c|c|}
\hline 
\multicolumn{1}{|c|}{Symbol} & Unicode & ASCII \\
\hline
tensor product & \Symbol{⊗} & \texttt{><} \\
\hline
\end{tabular}
\label{unicodetable}
\end{table}


%%\begin{abstract}
%%\noindent
%%
%%\end{abstract}

\bibliographystyle{plainnat}
\bibliography{bornat} 

\end{document}