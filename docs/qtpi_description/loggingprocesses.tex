% !TEX root = ./qtpi_description.tex

\chapter{Logging processes}
\chaplabel{loggingprocesses}

Before qtpi's trace mechanism was as advanced as it is now, I found it helpful to include commands that logged the effect of my programs. But logging information gets in the way of protocol descriptions. \Figref{teleportwithlogging}, for example, shows a teleport algorithm with logging information. The logging commands are intrusive and obscure the program: compare it with the simplicity of \figref{teleportfinal}.

\mvb{\teleportwithlogging}
proc System () = 
  (newq x=|+>, y=|0>) x,y>>CNot . 
  (new c:^(bit,bit)) | Alice(x,c) | Bob(y,c)

  Alice (x:qubit, c:^(bit,bit)) = 
    (newq z = |+>)  
    out!["\ninitially Alice's z is "] . 
    outq!(qval z) . out!["\n"] .
    z,x>>CNot . z>>H . z-/-(vz) . x-/-(vx) . 
    c!vz,vx . 

  Bob(y:qubit, c:^(bit,bit)) = 
    c?(b1,b2) . 
    y >> match (b1,b2) . + (0b0,0b0) . I
                         + (0b0,0b1) . X
                         + (0b1,0b0) . Z
                         + (0b1,0b1) . Z*X    .
    out!["finally Bob's y is "] . 
    outq!(qval y) . out!["\n"] .out!["finally Bob's y is "] . 
    outq!(qval y) . out!["\n"] . 
\end{myverbbox} %can't be a macro
\begin{figure}
\centering
\teleportwithlogging
\caption{A quantum teleport simulation, with logging statements}
\figlabel{teleportwithlogging}
\end{figure}

So I wondered if I could find a way of separating the logging commands from the program. This is what I devised: \emph{logging processes} which attach to the program at special \emph{logging points}, identified with a number. The logging point is shown with an insertion caret (⁁) plus the process identifier, which you can attach before or after a declaration or a process step. The logging processes attached to a process are written at the end of the process itself, preceded by \verbtt{with}. \Figref{teleportwithloggingprocesses} shows the same program as \figref{teleportwithlogging}, but with the logging separated out. A little bit clearer, at any rate. 

\mvb{\teleportwithloggingprocesses}
proc System () = 
  (newq x=|+>, y=|0>) x,y>>CNot . 
  (new c:^(bit,bit)) | Alice(x,c) | Bob(y,c)

  Alice (x:qubit, c:^(bit,bit)) = 
    (newq z = |+>) ⁁1
    z,x>>CNot . z>>H . z-/-(vz) . x-/-(vx) . 
    c!vz,vx . 
     
                                              with 1: out!["\ninitially Alice's z is "] . 
                                                      outq!(qval z) . out!["\n"] . 
  Bob(y:qubit, c:^(bit,bit)) = 
    c?(b1,b2) . 
    y >> match (b1,b2) . + (0b0,0b0) . I
                         + (0b0,0b1) . X
                         + (0b1,0b0) . Z
                         + (0b1,0b1) . Z*X    . ⁁1
     
                                                with 1: out!["finally Bob's y is "] . 
                                                        outq!(qval y) . out!["\n"] . 
\end{myverbbox} %can't be a macro
\begin{figure}
\centering
\teleportwithloggingprocesses
\caption{A quantum teleport simulation, with logging processes}
\figlabel{teleportwithloggingprocesses}
\end{figure}

A logging process acts like code inserted at the logging point, so it can use any names in scope at that point. But it can't 
\begin{itemize*}
\item create, gate, measure, send or receive a qubit;
\item receive classical messages;
\item contain any logging points;
\item invoke a process;
\item split into parallel subprocesses.
\end{itemize*}
So: not much really, apart from sending classical values down channels. Just what is needed to do a simple job of logging, and not taking over any of the protocol-description work of the process to which it is attached.

I don't think it's necessary to define the syntax formally.
