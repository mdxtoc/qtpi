\usepackage[braket]{qcircuit}

\newcommand {\cols}[1][*{50}{l}]{\begin{array}{#1}}
\newcommand {\sloc}{\end{array}}

\newcommand{\hstrut}[1]{\rule{#1}{0pt}}
\newcommand{\vstrut}[1]{\rule{0pt}{#1}}

\usepackage[ligature,reserved,shorthand]{semantic}
\newcommand{\varstyle}[1]{\ensuremath{\mathit{#1}}}
\reservestyle{\var}{\varstyle}
\newcommand{\bconststyle}[1]{\ensuremath{\mathbf{#1}}}
\reservestyle{\bconst}{\bconststyle}

%% syntax definitions
\newcommand{\wordstyle}[1]{\ensuremath{\operatorname{#1}}}
\reservestyle{\word}{\wordstyle}
\newcommand{\terminalstyle}[1]{\ensuremath{\mathrel{\mathsf{\mathop{#1}}}}}
\reservestyle{\terminal}{\terminalstyle}
\newcommand{\nonterminalstyle}[1]{\ensuremath{\mathit{#1}}}
\reservestyle{\nonterminal}{\nonterminalstyle}
\newcommand{\operatorstyle}[1]{\ensuremath{\operatorname{#1}}}
\reservestyle{\op}{\operatorstyle}

% mathligs must be arranged in reverse prefix order -- i.e. if prefixes match, longest symbol first
\mathlig{::=}{\mathrel{\mathop:\!\mathop:}=\ }

\newcommand{\nt}[1]{\setnonterm{#1}}
\newcommand{\term}[1]{\setterminal{#1}}

\newcommand{\alt}{\ \vert\ } % within a line
\newcommand{\Alt}{\hstrut{14pt}\mathpunct{\vert}\hstrut{5pt}} % beginning of line

\newcommand{\tdot}[0]{\text{.}}

\newcommand{\bv}[1]{\ensuremath{|#1\rangle}}
\newcommand{\zero}{\bv{\hspace{0.5pt}0}}
\newcommand{\one}{\bv{1}}
\newcommand{\plus}{\bv{\mathalpha{+}}}
\newcommand{\minus}{\bv{\mathalpha{-}}}

\newcommand{\zerozero}{\bv{\hspace{0.5pt}00}}
\newcommand{\onezero}{\bv{10}}
\newcommand{\zeroone}{\bv{\hspace{0.5pt}01}}
\newcommand{\oneone}{\bv{11}}

\newcommand{\zerozerozero}{\bv{\hspace{0.5pt}000}}
\newcommand{\zerozeroone}{\bv{\hspace{0.5pt}001}}
\newcommand{\zeroonezero}{\bv{\hspace{0.5pt}010}}
\newcommand{\zerooneone}{\bv{\hspace{0.5pt}011}}
\newcommand{\onezerozero}{\bv{100}}
\newcommand{\onezeroone}{\bv{101}}
\newcommand{\oneonezero}{\bv{110}}
\newcommand{\oneoneone}{\bv{one11}}

\newcommand{\vb}[1]{\ensuremath{\langle#1|}}

\newcommand\degree{\ensuremath{^{\circ}}}

\newcommand{\cvec}[2]{\ensuremath{\left(\begin{smallmatrix}#1\\#2\end{smallmatrix}\right)}}

\newcommand{\zerov}{\cvec{1}{0}}
\newcommand{\onev}{\cvec{0}{1}}
\newcommand{\plusv}{\cvec{h}{h}}
\newcommand{\minusv}{\cvec{h}{-h}}

\newcommand{\qgate}[4]{\ensuremath{\left(\begin{matrix}#1&#2\\#3&#4\end{matrix}\right)}}
\newcommand{\qgatetwo}[4]{\ensuremath{\left(\begin{matrix}#1\\#2\\#3\\#4\end{matrix}\right)}}

\newcommand{\smallqgate}[4]{\ensuremath{\left(\begin{smallmatrix}#1&#2\\#3&#4\end{smallmatrix}\right)}}
\newcommand{\smallqgatetwo}[4]{\ensuremath{\left(\begin{smallmatrix}#1\\#2\\#3\\#4\end{smallmatrix}\right)}}

\newcommand{\Xg}{\smallqgate{0}{1}{1}{0}}
\newcommand{\Ig}{\smallqgate{1}{0}{0}{1}}
\newcommand{\Zg}{\smallqgate{1}{0}{0}{-\!1}}
\newcommand{\Hg}{\smallqgate{h}{h}{h}{\;-\!h}}

\newcommand{\CNotg}{\ensuremath{\left(\begin{smallmatrix} 1 & 0 & 0 & 0 \\
							 							   0 & 1 & 0 & 0 \\
							                               0 & 0 & 0 & 1 \\
							                               0 & 0 & 1 &0 \\
                                \end{smallmatrix}\right)}}

\newcommand{\eqnlabel}[1]{\label{eqn:#1}}
\newcommand{\eqnref}[1]{(\ref{eqn:#1})}
\newcommand{\figlabel}[1]{\label{fig:#1}}
\newcommand{\figref}[1]{figure \ref{fig:#1}}
\newcommand{\Figref}[1]{Figure \ref{fig:#1}}
\newcommand{\seclabel}[1]{\label{sec:#1}}
\newcommand{\secref}[1]{section \ref{sec:#1}}
\newcommand{\Secref}[1]{Section \ref{sec:#1}}
\newcommand{\appxlabel}[1]{\label{appx:#1}}
\newcommand{\appxref}[1]{appendix \ref{appx:#1}}
\newcommand{\Appxref}[1]{Appendix \ref{appx:#1}}
\newcommand{\tablabel}[1]{\label{tab:#1}}
\newcommand{\tabref}[1]{table \ref{tab:#1}}
\newcommand{\Tabref}[1]{Table \ref{tab:#1}}
\newcommand{\deflabel}[1]{\label{def:#1}}
\newcommand{\defref}[1]{definition \ref{def:#1}}
\newcommand{\Defref}[1]{lemma \ref{def:#1}}
\newcommand{\lemlabel}[1]{\label{lem:#1}}
\newcommand{\lemref}[1]{lemma \ref{lem:#1}}
\newcommand{\Lemref}[1]{Lemma \ref{lem:#1}}

\word{CNot,Toffoli,T,I,Z,X,H}
