% !TEX root = ./qtpi_description.tex

\chapter{The symbolic calculator}
\chaplabel{calculator}
\chaplabel{symboliccalculator}

Apart from the deep dark mystery of quantum measurement, quantum computation is about matrix arithmetic using matrices, column and row vectors all containing complex numbers. When I began to work on qtpi, I wanted to avoid approximations, because I was afraid of accumulating calculation errors. Luckily GNU has produced a library called gmp that implements arithmetic on unbounded-precision rationals, and that avoids having to approximate, say, $1/3$. But that doesn't solve the whole problem: there are irrationals, notably square roots and trigonometric values, that crop up. And there are variables, as for example $a$ and $b$ in the teleportation example. 

After experimenting for a while with powers of square roots, I found a treatment based on trig functions and square roots.  

\section{The symbolic calculator}

A symbolic numeral is a sum of real part $x$ and imaginary part $y$, each a symbolic real, representing $x+iy$.

A symbolic real is a sum of symbolic real products. 

A symbolic real product is a rational (a fraction with unbounded integer numerator and denominator) and a sequence of elements. The elements are 
\begin{itemize*}
\item square roots of rationals;
\item single trigonometric functions $\sin{}$ and $\cos{}$ with rational arguments representing angles;
\item real and imaginary parts of variables $a_{i}$ and $b_{i}$.
\end{itemize*}

The rationals are provided by OCaml's Q package, itself an interface to the GNU multi-precision arithmetic library libgmp. The whole-number numerator and denominator in a rational are of unbounded length -- constrained, of course, by the physical quantities of memory of the computer running the library.  In practice, a usefully very very very large range.

Real and imaginary parts of a variable are shown, as infrequently as possible, as $a_{i}\mathbb{r}$ and $a_{i}\mathbb{i}$.

Arguments of trigonometric functions are interpreted as degrees (see \secref{degreesorradians}).

\subsection{Simplification}

Trigonemetric function arguments are reduced in construction to the interval $(0,45]$ (which doesn't include 0, but does include 45) by using the identities
\begin{itemize*}
\item $\cos\theta->\cos{(\theta\mod360)}$, $\sin\theta->\sin{(\theta\mod360)}$;
\item if $\theta>180$ then $\cos\theta-> \cos{(360-\theta)}$, $\sin\theta-> -\sin{(360-\theta)}$
\item if $\theta>90$ then $\cos\theta-> -\cos{(180-\theta)}$, $\sin\theta-> \sin{(180-\theta)}$
\item $\cos0->1$, $\sin90->1$, $\cos{90}->0$, $\sin0->0$
\item $\cos60->\frac{1}{2}$, $\sin30->\frac{1}{2}$
\item if $\theta>45$ then $\cos\theta->\sin{(90-\theta)}$, $\sin\theta->\cos{(90-\theta)}$
\item $\sin45->\cos45$
\end{itemize*}
 
$cos(45)$ is printed as $\frac{1}{\sqrt2}$, and $\frac{1}{\sqrt2}$ is replaced by $\cos{45}$; likewise $\frac{1}{\sqrt3}$ is represented by $\frac{2}{3}\cos{30}$.

Sums are simplified by:
\begin{itemize*}
\item $n*P+m*P=(n+m)*P$ -- sum the rationals if the element sequences are the same
\item $|a_{i}|^{2}+|b^{i}|^{2}$ -- if there are elements $n*P*a_{i}\mathbb{r}$, $n*P*a_{i}\mathbb{i}$, $n*P*b_{i}\mathbb{r}$, $n*P*b_{i}\mathbb{i}$, replace by $n*P$.
\end{itemize*}

Products are simplified by
\begin{itemize*}
\item $\sqrt{x}\times\sqrt{x}=x$
\item trigonometric identities, where $\theta\leq\phi$ (thanks to Wikipedia)
\begin{itemize*}
\item $2\cos\theta\cos\phi = \cos(\phi-\theta)+\cos(\theta+\phi)$
\item $2\sin\theta\sin\phi = \cos(\phi-\theta)-\cos(\theta+\phi)$
\item $2\sin\theta\cos\phi = \sin(\theta+\phi)-\sin(\phi-\theta)$
\item $2\cos\theta\sin\phi = \sin(\theta+\phi)+\sin(\phi-\theta)$
\end{itemize*}
\end{itemize*}

\subsection{Measurement}

The measurement mechanism calculates a probability from symbolic amplitudes, and to do so must approximate $\sin{}$, $\cos{}$ and $\sqrt{}$. That calculation also has access to the secret values of variables.

\section{Sparse-matrix optimisation}

One day I'll write about that. Functional representation of matrices allows huge gates to be represented: for example creating a W state with 1024 qubits requires a matrix of $2^{1024}*2^{1024}$ elements. But it's mostly a structure of functions, so takes up very little space.

\section{Constructing sxnum values}

Matrices and vectors contain complex numbers, pairs of symbolic reals, of type sxnum. Qtpi provides some useful constants and functions:
\begin{itemize}
\item \verb|sx_0|: sxnum, \verb|sx_1|: sxnum;
\item \verb|sx_i|: sxnum;
\item \verb|sx_sqrt|: num $->$ sxnum;
\item \verb|sx_sin|: num $->$ sxnum;
\item \verb|sx_cos|: num $->$ sxnum;
\end{itemize}
The arithmetic operators +, -, and * are overloaded to include sxnum $->$ sxnum $->$ sxnum. 

(Perhaps there should be an \verb|sx_num| function; I think I can't extend the arithmetic operators to include num $->$ sxnum $->$ sxnum and sxnum $->$ num $->$ sxnum.)
