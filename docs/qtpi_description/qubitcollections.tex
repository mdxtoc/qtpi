% !TEX root = ./qtpi_description.tex

\chapter{Qubit collections}
\chaplabel{qubitcollections}

Lots of algorithms work with arbitrary-sized groups of qubits. Grover's algorithm is an obvious example. The core notation of \chapref{corenotation} allows only single bits. Qubit collections fill the gap, but to preserve qtpi's guarantee of no sharing, no duplication there are some restrictions on their use.

\begin{itemize*}
\item There is a type \verbtt{qubits} (note the s).
\item There is a declaration $\verbtt{(newqs } x=Ex, \;...\; ,z=Ex )$. Names $x, ... ,z$ are of type \verbtt{qubits}, and the sizes of the collections they name are determined by the initialising expressions $Ex, ..., Ez$.
\item A collection can be gated \verbtt{qs>>>E} (note the triple chevron).
\item A collection can me measured \verbtt{qs⌢⃫(bpat)} (note the double meter needle); the result is a bit list, bound to names by the pattern bpat.
\item The elements of a qubit collection are qubits. They are indexed \verbtt{qs↓index}; indices are $0..n-1$ for an n-qubit collection. 
\begin{itemize*}
\item an element can be used, perhaps with other qubits, in a normal (\verbtt{>>}) gating step;
\item an indexed element can't be sent in a message, used as an argument in a process invocation, or measured.
\item if you measure an indexed element, the whole collection is measured.
\end{itemize*}
\item collections can be joined: $(\verbtt{joinqs } \setvar{qas}, ... , \setvar{qzs} → \setvar{qs})$. Each of \setvar{qas}, ... , \setvar{qzs} is a collection, used up in the joining (can't be used again, as if sent away); \setvar{qs} is their concatenation.
\item A collection can be split: $(\verbtt{splitqs } \setvar{qs} → \setvar{qas}(\setvar{ka}), ... , \setvar{qzs}(\setvar{kz}))$. The collection \setvar{qs} is used up; the new collections \setvar{qas}, ... , \setvar{qzs} are created, each of the length given by the lengths \setvar{ka}, ... , \setvar{kz}. The final length \setvar{kz} is optional.
\end{itemize*}

