% !TEX root = ./qtpi_description.tex

\chapter{Qubit collections and iteration}
\chaplabel{qubitcollections}

Lots of algorithms work with arbitrary-sized groups of qubits. Grover's algorithm is an obvious example; so is the W algorithm. The core notation of \chapref{corenotation} allows only single bits. \emph{Qubit collections} fill the gap, but to preserve qtpi's guarantee of `no sharing, no duplication' there are some restrictions on their use.

The original idea was that a collection of \setvar{n} qubits is a unit of resource: you create it as a whole, you send it as a whole, you receive it as a whole, and if you use \verbtt{measuredestroys} then you must measure it as a whole. Given those conditions, it is possible to use the same calculation as for single qubits to ensure no sharing and no duplication. 

\begin{itemize*}
\item There is a type \verbtt{qubits} (note the s).
\item There is a declaration $\verbtt{(newqs } x=Ex, \;...\; ,z=Ez )$. Names $x, ... ,z$ are of type \verbtt{qubits}, and the sizes of the collections they name are determined by the initialising expressions $Ex, ..., Ez$.
\item A collection can be gated \verbtt{qs>>>E} (note the triple chevron).
\item A collection can me measured \verbtt{qs⌢⃫(bpat)} (note the double meter needle); the result is a bit list, bound to names by the pattern bpat.
\item The elements of a qubit collection are qubits. They are indexed \verbtt{qs↓index}; indices are $0..n-1$ for an n-qubit collection. 
\begin{itemize*}
\item An element can be used, perhaps with other qubits, in a normal (\verbtt{>>}) gating step.
\item An element can't be sent in a message or used as an argument in a process invocation.
\item If you use \verb|-measuredestroys true| then you can't measure an indexed element.
\end{itemize*}
\item collections can be joined: $(\verbtt{joinqs } \setvar{qas}, ... , \setvar{qzs} → \setvar{qs})$. Each of \setvar{qas}, ... , \setvar{qzs} is a collection, used up in the joining (can't be used again, as if sent away); \setvar{qs} is their concatenation.
\item A collection can be split: $(\verbtt{splitqs } \setvar{qs} → \setvar{qas}(\setvar{ka}), ... , \setvar{qzs}(\setvar{kz}))$. The collection \setvar{qs} is used up; the new collections \setvar{qas}, ... , \setvar{qzs} are created, each of the length given by the lengths \setvar{ka}, ... , \setvar{kz}. The final length \setvar{kz} is optional.
\end{itemize*}

Note that indexing may cause an out-of-bounds error at run time. Note that gating multiple elements may cause an out-of-bounds error if the tuple of qubits is not disjoint.

\section{Iterative constructions}

I've probably overstepped the mark here: I'm not an inspired language designer. But it seems to me that iteration is not always best described by recursion: the mechanics overwhelms the algorithm. Grover's algorithm, for example, includes an iterative process: apply gating operations to a collection of qubits a predetermined number of times. \Figref{Groverrecursive} is a recursive version (definition of functions omitted). It creates \setvar{n} random bits and a collection of \setvar{n} random qubits, each set to \plus. Then it counts down recursively to perform the gating steps, and ends the recursion by measuring the qubits.

\mvb{\Groverrecursive}
  System () =
    . (let n = read_min_int 2 "number of bits")
    . (let bs = randbits n)
    . (newqs qs = |+>⊗⊗n)  

    . (let G = groverG n)
    . (let U = groverU bs)
    . (let iters = floor (𝝅*sqrt(2**n)/4+0.5))
    . out!["grover "; show n; " bs = "; show bs; "; "; show iters; " iterations"; "\n"]
    . Grover(G,U,qs,bs,n)
  
  Grover(G,U,qs,bs,n) =
    if n=0 then
      . out!["finally "] . outq!showq qs . out!["\n"]
        
      . qs⌢⃫(bs') 
      . out!["measurement says "; show bs'; 
               if bs=bs' then " ok" else " ** WRONG **"; "\n"] 
      . 
    else
      . qs>>>U . qs>>>G
      . Grover(G,U,qs,bs,n-1)

\end{myverbbox} %can't be a macro
\begin{figure}
\centering
\Groverrecursive
\caption{Grover's algorithm, expressed recursively}
\figlabel{Groverrecursive}
\end{figure}

An iterative step is a simple `for' loop in musical da capo brackets 𝄆 ... 𝄇 (where the closing bracket is, as usual, optional). It executes the process \setvar{P} with successive values of the list \setvar{L} bound to the name \setvar{x}.\footnote{Miranda enthusiasts will recognise this as a clumsy analogue of a list comprehension.}
\begin{equation*}
𝄆 x \ ← \ \setvar{list}\  : \ P \  𝄇
\end{equation*}

\Figref{Groveriterative} includes the iterative step
\mvb{\iterexample}
𝄆 i←tabulate iters (λ i. i): qs>>>U . qs>>>G . 
\end{myverbbox}
\begin{quote}
\iterexample
\end{quote}
and then continues with the steps that naturally follow the repetition. Much clearer than the recursive version, IMO.
\mvb{\Groveriterative}
  System () =
    . (let n = read_min_int 2 "number of bits")
    . (let bs = randbits n)
    . (newqs qs = |+>⊗⊗n)  

    . (let G = groverG n)
    . (let U = groverU bs)
    . (let iters = floor (𝝅*sqrt(2**n)/4+0.5))
    . out!["grover "; show n; " bs = "; show bs; "; "; show iters; " iterations"; "\n"]
    
    . 𝄆 i←tabulate iters (λ i. i): qs>>>U . qs>>>G . 
    
    . out!["finally "] . outq!showq qs . out!["\n"]
    
    . qs⌢⃫(bs') 
    . out!["measurement says "; show bs'; 
           if bs=bs' then " ok" else " ** WRONG **"; "\n"] 
    . 

\end{myverbbox} %can't be a macro
\begin{figure}
\centering
\Groveriterative
\caption{Grover's algorithm, expressed iteratively}
\figlabel{Groveriterative}
\end{figure}
